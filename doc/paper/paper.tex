\documentclass[pageno]{jpaper}

%replace XXX with the submission number you are given from the ISCA submission site.
\newcommand{\IWreport}{2015}

\usepackage[normalem]{ulem}
\usepackage{listings}
\usepackage{xcolor}
\lstset { %
    language=C++,
    backgroundcolor=\color{black!5}, % set backgroundcolor
    basicstyle=\ttfamily
}
\lstMakeShortInline[]|

\begin{document}

\title{Virtual Reality 3D Scanning}

\date{}
\maketitle

\thispagestyle{empty}

\begin{abstract}
  We propose a device and accompanying software that provides real-time feedback
  as a scene is 3D scanned. This method allows viewing the current scan from
  multiple angles and viewpoints, while also providing a feeling of presence
  through a virtual reality interface.
\end{abstract}

\section{Introduction}

\ldots

\section{Approach}

\ldots

\section{Implementation}

The program was written in C++, using C++11 features such as lambdas. It uses
OpenGL for rendering, SimpleDirectMedia Layer (SDL) to create a window and
maintain an OpenGL context, the Oculus Rift SDK to render to the Oculus Rift and
read position/rotation tracking data, the OpenGL Mathematics (GLM) library for
mathematical structures and operations and the Intel Double Springs 4 SDK to
read depth images from the Intel Double Springs 4 camera.

The implementation is divided into 4 main systems, each represented by a class:
|Game|, |VR|, |DS|, |Scene|. |Game| runs the main event loop, |VR| handles
virtual reality rendering, |DS| reads depth camera data and |Scene| manages
scene elements and draws them.

\subsection{\lstinline|Game|}

|Game| maintains an SDL window with an OpenGL context, listens for events and
notifies event handlers, and provides a frames-per-second display. The
constructor initializes SDL, creates a window and an OpenGL context, and the
destructor quits SDL.

In its public interface, the most important part of |Game| is the
|Game::update(update, draw)| method, which runs the main event loop, taking two
functions |update| and |draw|. The |update| function is called every loop, and
the |draw| function is called whenever the scene must be re-rendered. These are
separate callbacks to allow updating at a higher frequency than drawing, or to
allow update even when not drawing, such as when the window is minimized.

Other than this, |Game| provides methods to retrieve the SDL window, access
command-line arguments and quit the program.

\subsection{\lstinline|VR|}

|VR| is the main interface to the virtual reality display. It uses the Oculus
Rift SDK to communicate with the Oculus Rift device, and the implementation has
been designed so there are no references to the Oculus Rift SDK elsewhere in the
code and where required the device is communicated with only through the |VR|
class. The constructor creates an OpenGL framebuffer to render to that is shared
with the Oculus SDK, resizes and repositions the window to render to the Oculus
Rift, and enables position and rotation tracking. The destructor deinitializes
all of this.

The public |VR::draw(drawer)| method draws to the virtual reality display,
taking a function |drawer| that draws the actual scene as if drawing to a
conventional screen. It achieves this by calling |drawer| twice, once for each
eye, each time setting up the projection and view matrices according to the eye
rendered from. The view matrix is dependent upon the position and rotation of
the head-mounted display and takes into account the displacement of each eye
from the center of the head. To calculate the view matrix it computes the
inverse of the matrices returned by |VR::eye_transforms()|.

The public |VR::eye_transforms()| returns two matrices in an |std::array|, each
representing the camera matrix of an eye. It also supports an optional boolean
parameter |mid|. If |mid| is true, it returns the camera matrix of the `mid' eye,
as if viewing the scene from the center of the head with the same eye
orientation.

\subsection{\lstinline|DS|}

|DS| is the main interface to the depth camera. It uses the Intel Double Springs
4 SDK to communicate with the Intel Double Springs 4 depth camera. The
constructor probes for the depth camera configuration, enables Z (depth) capture
and sets the camera resolution to 480x360.

The public |DS::points()| method captures a depth image from the camera and
returns an |std::vector| representing a point cloud. These points are placed in
the camera-space of the depth camera, with the $z$-axis pointing forward, the
$x$ axis pointing right and the $y$ axis pointing up, in a frame that looks
`down' through the camera.

\subsection{\lstinline|Scene|}

|Scene| handles the maintenance and drawing of scene elements. In the current
implementation, there are no scene elements other than the point cloud
itself. The scene can be rotated and zoomed in and out of with the mouse. The
|Scene::update()| function must be called every update to handle mouse
events. |Scene::draw(points, vr)| draws the point cloud, taking the array of
points |points| and the current |VR| instance |vr|. The |VR| instance is
required so that the points can be placed in their correct world-space position
using the `mid' eye camera matrix returned by |VR::eye_transforms()|. Since the
depth camera is attached to the front of the head-mounted display, this achieves
the effect of registering multiple captures from the depth camera.

\ldots

\bstctlcite{bstctl:etal, bstctl:nodash, bstctl:simpurl}
\bibliographystyle{IEEEtranS}
\bibliography{references}

\end{document}


%%% Local Variables:
%%% mode: latex
%%% TeX-master: t
%%% End:
